\documentclass[letterpaper]{article} 
\usepackage[utf8]{inputenc}
\usepackage[T1]{fontenc}
\usepackage{amsmath}
\usepackage{amsfonts}
\usepackage{amssymb}
\usepackage{array}
\usepackage{booktabs}
\usepackage{hyperref}
\usepackage[version=4]{mhchem}
\usepackage{stmaryrd}
\usepackage[dvipsnames]{xcolor}
\colorlet{LightRubineRed}{RubineRed!70}
\colorlet{Mycolor1}{green!10!orange}
\definecolor{Mycolor2}{HTML}{00F9DE}
\usepackage{graphicx}
\usepackage{amsmath}
\usepackage{graphicx}
\usepackage{capt-of}
\usepackage{lipsum}
\usepackage{fancyvrb}
\usepackage{tabularx}
\usepackage{listings}
\usepackage[export]{adjustbox}
\graphicspath{ {./images/} }
\usepackage[utf8]{inputenc}
\usepackage[english]{babel}
\usepackage{float}
\usepackage{lipsum}
\usepackage{graphicx}
\usepackage{float}
\usepackage[margin=0.7in]{geometry}
\usepackage{amsmath}
\usepackage{graphicx}
\usepackage{capt-of}
\usepackage{tcolorbox}
\usepackage{lipsum}
\usepackage{graphicx}
\usepackage{float}
\usepackage{listings}
\usepackage{hyperref} 
\usepackage{xcolor} % For custom colors
\lstset{
	language=Python,                % Choose the language (e.g., Python, C, R)
	basicstyle=\ttfamily\small, % Font size and type
	keywordstyle=\color{blue},  % Keywords color
	commentstyle=\color{gray},  % Comments color
	stringstyle=\color{red},    % String color
	numbers=left,               % Line numbers
	numberstyle=\tiny\color{gray}, % Line number style
	stepnumber=1,               % Numbering step
	breaklines=true,            % Auto line break
	backgroundcolor=\color{black!5}, % Light gray background
	frame=single,               % Frame around the code
}
\usepackage{float}
\usepackage[]{amsthm} %lets us use \begin{proof}
	\usepackage[]{amssymb} %gives us the character \varnothing
	
	\title{Homework 1, MATH 5398}
	\author{Zongyi Liu}
	\date{Wed, Oct 1, 2025}
	\begin{document}
		\maketitle
		
		\section{Preparations}
		\subsection{Data Sources}
		
		Firstly we are asked to retrieve data from three sources:
		
		\begin{itemize}
			\item \href{https://github.com/fja05680/sp500}{ S\&P 500 Historical Components \& Changes}
			\item S\&P 500 component stocks from 1996-01-01 to the most current date in \href{https://wrds-www.wharton.upenn.edu/login/?next=/pages/get-data/compustat-capital-iq-standard-poors/compustat/north-america-daily/fundamentals-quarterly/}{WRDS-Fundamentals Quarterly}
			\item daily data for S\&P 500 component stocks from 1996-01-01 to the most current date from \href{https://wrds-www.wharton.upenn.edu/login/?next=/pages/get-data/compustat-capital-iq-standard-poors/compustat/north-america-daily/security-daily/}{WRDS-Security Daily}
		\end{itemize}		
		
		\subsection{Data Processing}
		
		Then we use the provided \texttt{Step2\_preprocess\_fundmental\_data.py}
		to process two files, where we generated a final result in \texttt{final\_ratios.csv}, which contains headers like PE (Price–to-Earnings Ratio), PS (Price-to-Sales Ratio), PB (Price-to-Book Ratio), OPM (Operating Margin), NPM (Net Profit Margin), ROA (Return On Assets), ROE (Return on Equity), EPS (Earnings Per Share), BPS (Book Per Share), DPS (Dividend Per Share), etc. 
		
		In using FinRL Trading model, we follow the dynamic stock recommendation framework proposed by Yang, Liu, and Wu (2018) \cite{yang2018practical}., which formulates stock selection as a rolling, cross-sectional machine learning ranking problem rather than static factor-based prediction.
		
		There are also 11 excel files inside the folder, showing 11 different sectors of stocks provided in the S\&P 500.
		
		
		
		
		\section{Mean--Variance Portfolio Construction}
		
		\subsection{Theoretical Framework}
		
		Consider $n$ risky assets with expected returns 
		$\boldsymbol{\mu} = (\mu_1, \mu_2, \ldots, \mu_n)^\top$ 
		and covariance matrix $\Sigma \in \mathbb{R}^{n \times n}$.  
		Let $\boldsymbol{w} = (w_1, w_2, \ldots, w_n)^\top$ denote the portfolio weights, satisfying $
		\sum_{i=1}^{n} w_i = 1.$
		
		\noindent Then\cite{markowitz1952portfolio}:
		
		\[
		\text{Expected return: } \quad 
		\mu_p = \boldsymbol{w}^\top \boldsymbol{\mu},
		\]
		\[
		\text{Portfolio variance: } \quad 
		\sigma_p^2 = \boldsymbol{w}^\top \Sigma \boldsymbol{w}.
		\]
		
		\subsection{Optimization Problems}
		
		\underline{ Minimum-Variance Portfolio}
		
		Given a target return $r^*$, the investor solves:
		\[
		\begin{aligned}
			\min_{\boldsymbol{w}} \quad & \boldsymbol{w}^\top \Sigma \boldsymbol{w} \\
			\text{s.t.} \quad 
			& \boldsymbol{w}^\top \boldsymbol{\mu} = r^*, \\
			& \boldsymbol{w}^\top \mathbf{1} = 1.
		\end{aligned}
		\]
		
		\underline{Tangency Portfolio}
		
		Assuming a risk-free rate $r_f$:
		\[
		\max_{\boldsymbol{w}} \frac{\boldsymbol{w}^\top (\boldsymbol{\mu} - r_f \mathbf{1})}{\sqrt{\boldsymbol{w}^\top \Sigma \boldsymbol{w}}}.
		\]
		
		
		\underline{Key Formulas}
		
		\[
		E[R_p] = \boldsymbol{w}^\top \boldsymbol{\mu}, 
		\qquad 
		\sigma_p^2 = \boldsymbol{w}^\top \Sigma \boldsymbol{w}.
		\]
		
		\[
		\text{Sharpe ratio: } 
		\quad S_p = \frac{E[R_p] - r_f}{\sigma_p}.
		\]
		
		
		\underline{ Efficient Frontier}
		
		The efficient frontier represents all optimal portfolios that minimize variance for a given expected return:
		\[
		\mathcal{F} = \{ (\sigma_p, \mu_p) : 
		\exists \boldsymbol{w} \text{ s.t. } 
		\mu_p = \boldsymbol{w}^\top \boldsymbol{\mu}, \;
		\sigma_p^2 = \boldsymbol{w}^\top \Sigma \boldsymbol{w} \}.
		\]
		
		\subsection{Remarks}
		
		\begin{itemize}
			\item If short selling is prohibited, impose $w_i \ge 0$.
			\item Adding a risk-free asset allows the construction of the Capital Market Line (CML).
			\item The tangency portfolio gives the highest Sharpe ratio and serves as the optimal risky portfolio.
		\end{itemize}
		
		Here we used the \texttt{minimize} in \texttt{scipy} to calculate the MVP. 
		
		\section{Backtesting}
		
		Then I used the \texttt{backtest.py} file to do the back testing from the year 2018 as the assignment suggested. 
		
		
		\includegraphics[max width=\textwidth, center]{Backtest}
		
		
		\begin{table}[htbp]
			\centering
			\caption{Performance Metrics of Portfolio and Market Benchmarks (2018--2025)}
			\label{tab:performance_comparison}
			\begin{tabular}{lccc}
				\toprule
				Metric & Portfolio & S\&P 500 & QQQ \\
				\midrule
				Cumulative Return (\%)   & 1289.42 & 136.86 & 272.96 \\
				Annual Return (\%)      & 13.21   & 13.77  & 18.51  \\
				Annual Volatility (\%)  & 16.48   & 16.74  & 19.95  \\
				Max Drawdown (\%)       & -27.95  & -24.77 & -32.58 \\
				Sharpe Ratio            & 0.8017  & 0.6389 & 0.8053 \\
				Win Rate (\%)           & 55.12   & --     & --     \\
				Information Ratio       & 0.2984  & --     & --     \\
				\bottomrule
			\end{tabular}
		\end{table}
		
		As shown in table, the portfolio significantly
		outperforms the S\&P~500 and QQQ in cumulative terms, achieving a return of
		1289.42\% over the 2018--2025 period. Despite this strong long-term growth, the
		portfolio maintains an annual return (13.21\%) comparable to the S\&P~500 while
		exhibiting lower annual volatility (16.48\%), resulting in a high Sharpe ratio of
		0.8017.
		
		The portfolio’s maximum drawdown of $-27.95\%$ lies between that of the S\&P~500
		and QQQ, indicating a balanced exposure to growth and downside risk. Moreover, a
		win rate of 55.12\% and a positive information ratio of 0.2984 suggest that excess
		returns are generated in a stable and systematic manner. Overall, the results
		highlight a portfolio that delivers strong risk-adjusted performance while
		controlling volatility and drawdowns.
		
		
		
		
		\bibliographystyle{unsrt}
		\bibliography{ref}
		
	\end{document}